During geomagnetic storms, terawatts of power in the million mile-per-hour solar wind pierce the Earth’s magnetosphere. Geomagnetic storms and substorms create transverse magnetic waves known as Alfvén waves. 
In the auroral acceleration region, Alfvén waves accelerate electrons up to one-tenth the speed of light via wave-particle interactions. 
These inertial Alfvén wave (IAW) accelerated electrons are imbued with sub-100 meter structure perpendicular to geomagnetic field B. The IAW electric field parallel to B accelerates electrons up to about 10 keV along B. 
The IAW dispersion relation quantifies the precipitating electron striation observed with high-speed cameras as spatiotemporally dynamic fine structured aurora.

A network of tightly synchronized tomographic auroral observatories using model based iterative reconstruction (MBIR) techniques were developed in this dissertation. 
The TRANSCAR electron penetration model creates a basis set of monoenergetic electron beam eigenprofiles of auroral volume emission rate for the given location and ionospheric conditions. 
Each eigenprofile consists of nearly 200 broadband line spectra modulated by atmospheric attenuation, bandstop filter and imager quantum efficiency. 
The L-BFGS-B minimization routine combined with sub-pixel registered electron multiplying CCD video stream at order 10 ms cadence yields estimates of electron differential number flux at the top of the ionosphere. 

Our automatic data curation algorithm reduces one terabyte/camera/day into accurate MBIR-processed estimates of IAW-driven electron precipitation microstructure. 
This computer vision structured auroral discrimination algorithm was developed using a multiscale dual-camera system observing a \unit[175]{km} and \unit[14]{km} swath of sky simultaneously. 
This collective behavior algorithm exploits the “swarm” behavior of aurora, detectable even as video SNR approaches zero. 
A modified version of the algorithm is applied to topside ionospheric radar at Mars and broadcast FM passive radar. 
The fusion of data from coherent radar backscatter and optical data at order \unit[10]{ms} cadence confirms and further quantifies the relation of strong Langmuir turbulence and streaming plasma upflows in the ionosphere with the finest spatiotemporal auroral dynamics associated with IAW acceleration. 
The software programs developed in this dissertation solve the century-old problem of automatically discriminating finely structured aurora from other forms and pushes the observational wave-particle science frontiers forward.

