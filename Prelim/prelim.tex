% !TEX root = thesis.tex

\title{Alfvén Waves Underlying Ionospheric Destabilization: Ground-Based Observations}

\author{Michael Hirsch}

% Type of document prepared for this degree:
%   1 = Master of Science thesis,
%   2 = Doctor of Philisophy dissertation.
%   3 = Master of Science thesis and Doctor of Philisophy dissertation.
\degreetype=2

\prevdegrees{B.S. (High Honors), Michigan State University, 2009\\
	M.S., Boston University, 2014}

\department{Department of Electrical and Computer Engineering}

% Degree year is the year the diploma is expected, and defense year is
% the year the dissertation is written up and defended. Often, these
% will be the same, except for January graduation, when your defense
% will be in the fall of year X, and your graduation will be in
% January of year X+1
\defenseyear{2017}
\degreeyear{2017}

% For each reader, specify appropriate label {First, Second, Third},
% then name, and title. IMPORTANT: The title should be:
%   "Professor of Electrical and Computer Engineering",
% or similar, but it MUST NOT be:
%   Professor, Department of Electrical and Computer Engineering"
% or you will be asked to reprint and get new signatures.
% Warning: If you have more than five readers you are out of luck,
% because it will overflow to a new page. You may try to put part of
% the title in with the name.
\reader{First}{Joshua L. Semeter, Ph.D.}{Professor of Electrical and Computer Engineering}
\reader{Second}{W. Clem Karl, Ph.D.}{Professor and Chair of Electrical and Computer Engineering\\
Professor of Systems Engineering\\
Professor of Biomedical Engineering}
\reader{Third}{Michael J. Mendillo, Ph.D.}{Professor of Astronomy\\
	Professor of Electrical and Computer Engineering}
\reader{Fourth}{Mark N. Horenstein, Ph.D.}{Professor of Electrical and Computer Engineering}

% The Major Professor is the same as the first reader, but must be
% specified again for the abstract page. Up to 4 Major Professors
% (advisors) can be defined. 
\numadvisors=1
\majorprof{Joshua L. Semeter, Ph.D.}{{Professor of Electrical and Computer Engineering}}


%%%%%%%%%%%%%%%%%%%%%%%%%%%%%%%%%%%%%%%%%%%%%%%%%%%%%%%%%%%%%%%%  

%                       PRELIMINARY PAGES
% According to the BU guide the preliminary pages consist of:
% title, copyright (optional), approval,  acknowledgments (opt.),
% abstract, preface (opt.), Table of contents, List of tables (if
% any), List of illustrations (if any). The \tableofcontents,
% \listoffigures, and \listoftables commands can be used in the
% appropriate places. For other things like preface, do it manually
% with something like \newpage\section*{Preface}.

% This is an additional page to print a boxed-in title, author name and
% degree statement so that they are visible through the opening in BU
% covers used for reports. This makes a nicely bound copy. Uncomment only
% if you are printing a hardcopy for such covers. Leave commented out
% when producing PDF for library submission.
%\buecethesistitleboxpage

% Make the titlepage based on the above information.  If you need
% something special and can't use the standard form, you can specify
% the exact text of the titlepage yourself.  Put it in a titlepage
% environment and leave blank lines where you want vertical space.
% The spaces will be adjusted to fill the entire page.
\maketitle
\cleardoublepage

% The copyright page is blank except for the notice at the bottom. You
% must provide your name in capitals.
\copyrightpage
\cleardoublepage

% Now include the approval page based on the readers information
\approvalpage
\cleardoublepage

%\newpage
%\input{Prelim/quote}

% The acknowledgment page should go here. Use something like
% \newpage\section*{Acknowledgments} followed by your text.
%\section*{\centerline{Acknowledgments}}
%\input{Prelim/ack}
%\cleardoublepage

% The abstractpage environment sets up everything on the page except
% the text itself.  The title and other header material are put at the
% top of the page, and the supervisors are listed at the bottom.  A
% new page is begun both before and after.  Of course, an abstract may
% be more than one page itself.  If you need more control over the
% format of the page, you can use the abstract environment, which puts
% the word "Abstract" at the beginning and single spaces its text.

\begin{abstractpage}
During geomagnetic storms, terawatts of power in the million mile-per-hour solar wind pierce the Earth’s magnetosphere. Geomagnetic storms and substorms create transverse magnetic waves known as Alfvén waves. 
In the auroral acceleration region, Alfvén waves accelerate electrons up to one-tenth the speed of light via wave-particle interactions. 
These inertial Alfvén wave (IAW) accelerated electrons are imbued with sub-100 meter structure perpendicular to geomagnetic field B. The IAW electric field parallel to B accelerates electrons up to about 10 keV along B. 
The IAW dispersion relation quantifies the precipitating electron striation observed with high-speed cameras as spatiotemporally dynamic fine structured aurora.

A network of tightly synchronized tomographic auroral observatories using model based iterative reconstruction (MBIR) techniques were developed in this dissertation. 
The TRANSCAR electron penetration model creates a basis set of monoenergetic electron beam eigenprofiles of auroral volume emission rate for the given location and ionospheric conditions. 
Each eigenprofile consists of nearly 200 broadband line spectra modulated by atmospheric attenuation, bandstop filter and imager quantum efficiency. 
The L-BFGS-B minimization routine combined with sub-pixel registered electron multiplying CCD video stream at order 10 ms cadence yields estimates of electron differential number flux at the top of the ionosphere. 

Our automatic data curation algorithm reduces one terabyte/camera/day into accurate MBIR-processed estimates of IAW-driven electron precipitation microstructure. 
This computer vision structured auroral discrimination algorithm was developed using a multiscale dual-camera system observing a \unit[175]{km} and \unit[14]{km} swath of sky simultaneously. 
This collective behavior algorithm exploits the “swarm” behavior of aurora, detectable even as video SNR approaches zero. 
A modified version of the algorithm is applied to topside ionospheric radar at Mars and broadcast FM passive radar. 
The fusion of data from coherent radar backscatter and optical data at order \unit[10]{ms} cadence confirms and further quantifies the relation of strong Langmuir turbulence and streaming plasma upflows in the ionosphere with the finest spatiotemporal auroral dynamics associated with IAW acceleration. 
The software programs developed in this dissertation solve the century-old problem of automatically discriminating finely structured aurora from other forms and pushes the observational wave-particle science frontiers forward.


\end{abstractpage}
\cleardoublepage

% Now you can include a preface. Again, use something like
% \newpage\section*{Preface} followed by your text

% Table of contents comes after preface
\tableofcontents
\cleardoublepage

% If you do not have tables, comment out the following lines
\newpage
\listoftables
\cleardoublepage

% If you have figures, uncomment the following line
\newpage
\listoffigures
\cleardoublepage

% List of Abbrevs is NOT optional (Martha Wellman likes all abbrevs listed)
\chapter*{List of Abbreviations}
\begin{center}
  \begin{tabular}{lll}
    \hspace*{2em} & \hspace*{1in} & \hspace*{4.5in} \\
    DMC & \dotfill & Dual Multiscale Camera \\
    DSAD & \dotfill & Dynamic Structured Aurora Discriminator \\
    EMCCD  & \dotfill & Electron-Multiplying Charge Coupled Device \\
    HiST & \dotfill & High Speed Tomography System \\
    sCMOS & \dotfill & scientific Complementary Metal Oxide Semiconductor \\
    VER & \dotfill & volume emission rate \\


  \end{tabular}
\end{center}
\cleardoublepage

% END OF THE PRELIMINARY PAGES

\newpage
\endofprelim
