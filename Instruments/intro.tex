\section{Auroral Observation Background}\label{sec:obshistory}
The earliest dedicated attempts to photograph aurora in the First Polar Year 1882--1883 was unsuccessful despite \unit[4]{min} exposures \citep{egeland2013}.
The first actual photograph of aurora in 1892 was of too poor quality for quantitative use \citep{egeland2013}.
J. Sýkora's spectrographic plates collected in 1899--1900 were the first quality spectrographic data \citep{chernouss2008}. 
Birkeland tried from 1898--1900 unsuccessfully to simultaneously photograph aurora with two cameras separated by \unit[3.4]{km}, and in 1910 Størmer failed to estimate auroral heights with \unit[4.3]{km} separation \citep{egeland2013} (NB: HiST phase 1 camera separation \unit[3.1]{km}).
Størmer's contributions from 1909 onward resulted in his \unit[0.2]{s} exposure camera carefully constructed to record wavelengths from infrared to UV \citep{stormer1932} becoming the standard camera for IPY 1932.
Hundreds of Størmer's cameras collected a half-million photos, of which tens of thousands were of science-quality \citep{egeland2013}.
Størmer's emphasis on collecting prompt auroral emissions with careful star-based plate scale registration from multiple synchronized cameras separated by about $\unit[20..100]{km}$ may be thought of as direct philosophical ancestors to the HiST system developed during the course of this dissertation.
The problems Størmer's team faced in parsing 500,000 images would have been alleviated by the auroral discriminator developments in chapter~\ref{chapter:discrim}.
Størmer noted that processing one night's image stacks took about a month to process \citep{egeland2013}.

Despite the numerous advances during his long life by himself and others in auroral observations, Størmer passed away before unsolved magnetospheric and ionospheric puzzles began to be resolved via \textit{in situ} sensing of the early satellites.
A 1953 crowdsourcing auroral observation effort using prepaid postcards was plagued by imprecision and inaccuracy of human sighted auroral feature az/el.
It was recognized that replicable, precisely registered, synchronized wide-field optical data was vital to understanding auroral correlation to ionospheric perturbations \citep{birkeland1908,stormer1930,stoffregen1955}.
Leading up to IGY 1957, a network of newly designed ``all-sky'' cameras synchronized with broadband HF receivers were deployed in and near the auroral oval region in Scandinavia.
This effort was designed for better quantification of auroral behavior vs. ionospheric perturbations known then to affect LF-VHF interactions in the ionosphere.
The stations, manufactured in $2\times2$~m huts used the \unit[50]{Hz} power grid via synchronous film-driving motors to ensure sufficiently tight synchronization of the all-sky video. 
A typical setup was \unit[7]{s} uniform exposure, chosen to avoid overexposure while still capturing weak aurora on the \unit[100]{line/mm} film, at \unit[1]{min.} cadence.
Although temporal aliasing was clearly evident in the film strips, this aliasing was a compromise chosen for the great cost of the film required for months to years of observation.
The star registration vital to making altitude estimates of aurora were based on methods by \citet{stormer1930} as updated using current technology.
Deployment of all-sky cameras more extensively about the north and south auroral ovals led quickly to conclusions on the geomagnetic alignment of aurora and the diurnal behavior of the oval versus solar zenith angle \citep{denholm1961}.
The THEMIS all-sky camera network across North America has provided numerous new geophysical insights as used with its satellite--all-sky pairing as well as ISR and rocket experiments \citep{donovan2006}.
The upcoming frame rate improvements via the tREX system \citep{liang2016} will provide a new generation of insights from a medium frame-rate imaging network added to the continent-wide THEMIS ASI network through Canada and Alaska.
HiST phase 2 deployments may include coördinated observations with the tREX/THEMIS network.

A great challenge across STEM disciplines in general and the geosciences specifically is the storage and processing of vast amounts of data over constrained CPU and data bandwidth resources.
The first digital image scanner in 1957 could not hold the 2 bit, $176 \times 176$ pixel image in its \unit[6]{kB} memory, and so the persistence of a scanning oscilloscope phosphor provided the assembled image for human visual appreciation \citep{kirsch1958}.
Over a decade later, the forefront of automated microscope slide processing was an extensively customized computer constructed in a collaboration between multiple US Federal agencies.
This ``Spectre II'' automated cytometry machine was capable of $256 \times 256$ pixel 8-bit resolution, with a single image unable to fit in the \unit[16]{kB} memory \citep{shapiro1969}.
The data reduction employed by Spectre II overcame the issue of a full digitized multi-chromatic microscope slide consuming \unit[62.5]{GB} of storage.
At the time that amount of tape storage would have been over \unit[1200]{km} in length \citep{shapiro1969}, stretching from the earth's surface to over the top of the ionosphere if held vertical.
A contemporary IBM System/360 Model 91 mainframe used by NASA for the Apollo missions had a \unit[16]{MHz} CPU and \unit[2]{MB} of RAM \citep{ibm1967}.
Such computing power was not available in commodity desktop PC form until the late 1980s.
It was not until the 1990s that computer-controlled auroral observation systems with digital storage became common \citep{bjornthesis}.
Computers and hard drives after 2010 (see section~\ref{sec:prochistory}) were finally fast enough to sustain all-night operations from a full-frame, \unit[50]{fps} camera, a key requirement for distinguishing the acceleration mechanism behind plasma turbulence associated with multiple particular auroral forms.

CCD cameras began integration into astronomy in the 1970s \citep{lynds1975} and gradually digital imaging began taking over nearly all aspects of science and personal photography over the next three decades as challenges of imaging efficiency were resolved \citep{monet1993}.
In the past decade, the value of scientific CMOS (sCMOS) cameras has been proven for the high resolution high speed imaging. 
Cooled sCMOS cameras generally exceed the performance of CCD cameras for many applications.
In the most photon-starved regimes, EMCCD cameras reign supreme, as in the HiST system, which like Størmer's system a century before targets prompt emissions from UV to IR.
