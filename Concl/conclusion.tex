\chapter{Conclusions and Future Work}
\label{chapter:Conclusions}
\thispagestyle{myheadings}
\graphicspath{{Concl/Figures/}}

\setlength{\epigraphwidth}{0.85\textwidth}
\epigraph{To the astronomer, after the celestial body has disappeared from his view, his main work starts.}{Gauss 1839 as translated by \citet{gauss1839}}

This dissertation presents the evolution of auroral morphology observations from antiquity and particularly from 1900 through the critical advances made by the dissertation work.
In particular, the advances in personal computing technology of the past decade were vital to enabling frame rates $\gg \unit[10]{fps}$ for more than five seconds at a manual push as discussed in chapter~\ref{chapter:inst}.
More importantly, the ability to sustain \unit[20]{ms} video cadence for months of unattended operation was developed in chapter~\ref{chapter:discrim}.
The instrumental and algorithm advancements by this dissertation work allow researchers to ``set and forget'' their cameras, allowing the capture of rare and infrequent auroral and meteor events and other as yet poorly characterized phenomena.
This may someday help solve new problems such as the theorized particle acceleration due to inertial Alfvén resonators and distinguish between the numerous inertial Alfvén wave modes via optical emissions due to the particles accelerated by the waves.

This dissertation presents work that unlocked several vital capabilities for recording, curating and processing into science quantities such high speed auroral data.
In chapter~\ref{chapter:discrim}, the ability to discriminate spatiotemporally fine-scale aurora from other types of aurora was demonstrated, eliminating the vast majority of auroral video recorded.
From the video segments that remained, in chapter~\ref{chapter:sim} a model-based iterative reconstruction algorithm incorporating a physics based electron penetration model of the ionosphere connected camera to radar data in joint ISR-Optical analysis and characterization of driving forces behind particular dynamic auroral spatiotemporal morphologies.
Chapter~\ref{chapter:fusion} presented real data inversions for HiST in light of ISR data, coming up with a joint quantitative solution based on high speed data from both instruments.

The algorithms created are immediately available to the public via Github and have already been used by other geoscience researchers studying auroral and other phenomenon. 
Algorithmic and science contributions from other geoscientists have already occurred with the Fortran and Python code developed during this dissertation.
Stable versions of the code are archived at Zenodo and assigned DOI for scholarly archival purposes.
As funding and time permit, the instruments and algorithms will be used and expanded for future projects.

A corollary to the dissertation work has been Michael Hirsch's outreach and advocacy at the state and federal legislatures in support of geoscience and geospace research.
The number of constituent phone calls, letters and emails legislative offices receive concerning the priority of basic science is said by their office staff to be nearly nil.
Taking a PiRadar prototype to Capitol Hill, where numerous US Senators and Representatives have Ph.D. staffers to assist the Members in their legislative and policy decisions engendered significant interest.
In reviewing the historical geoscience efforts described in chapters~\ref{chapter:intro} and~\ref{chapter:physics}, it is apparent that finding ways to involve pre-college students directly in geoscience STEM outreach and research is at least a century-old tradition.
The complete permeation of computer technology in geoscience only makes such efforts more feasible, and finding commercial synergies in space weather research provides a significant catalyst to traditional geoscience efforts, particularly instrumental efforts where numerous inexpensive field sensors are deployed.

\section{Auroral Kinetics Model-Based Iterative Reconstruction}
A linear basis set using eigenprofiles generated by TRANSCAR was used along with L-BFGS-B minimization to rapidly find estimates of differential number flux at \unit[20]{ms} cadence. 
This eigenprofile minimization technique allows solving ill-posed, ill-conditioned systems for which no other solution method had yet been uncovered.
The hundreds of spectral lines involved makes far more efficient use of the sparse prompt auroral emissions, enabling frame rates beyond \unit[50]{fps} with the latest generation EMCCD cameras.
Over 100 years of auroral stereographic and tomographic observations (and, auroral observations in general) have been plagued by engineering limitations of the cameras.
This dissertation in chapters~\ref{chapter:inst} and~\ref{chapter:discrim} solved at least a substantial subset of those problems for high speed auroral video, opening the door to inexpensive (in human time and hardware) networks of high speed auroral tomography systems.
Now that data can be inverted at the fastest time scales possible, right up to the limits presented by the physics of the geomagnetic transmission line dispersion, further advances in ISR plasma line measurements will allow independent verification and data fusion of ionization measurements.
This will allow improved high resolution time-dependent modeling of ionospheric energy deposition and plasma flows, of which few models beyond the TRANSCAR model exist.

\section{Structured Auroral Discriminator}
No more should auroral researchers have to sit in cold sheds waiting to press record.
Hard drives are inexpensive and after discrimination and selection of video having desired auroral traits, a stack of USB HDD connected to a USB hub can hold a season's worth of data.
Sub-\$1000 PCs are sufficient to handle the workload, and the free open-source code used throughout means that one need not bother with the cost and headaches of Windows or Mac operating systems (although the code works on Mac/Windows as well as Linux).
The collective behavior algorithm employed is general enough that with slight modification it was used for passive radar and MARSIS HF radar as described in the respective appendices.
Upcoming software defined passive radars will benefit from these algorithms, as the data rate (and RF bandwidth) will be far higher than the system treated in appendix~\ref{chapter:passive}.

\section{Joint Optical and ISR Analysis}
ISR can measure at a cadence less than \unit[20]{ms}, with EMCCD cameras also capable of sub-\unit[20]{ms} measurements.
The implicit connection between prompt auroral emissions and ionization processes makes the pair of sensors a natural for data fusion.
This dissertation showed confirming results of the association between Alfvénic aurora of multiple types and Langmuir turbulence.
The next deployment of HiST will be associated with targeted observation modes taking the best advantage of the three camera system.
The ANDESITE mission launching in autumn 2017 will be a natural for joint ISR-optical measurements with the fine $B$-field measurements enabled by the ANDESITE Cubesat constellation.

\section{Autonomous Auroral Outposts}
Environmentally robust with a single connection to the outside world (120Vac power), the HiST phase 2 cabinets hold two cameras, two PCs and additional hardware for GPS and beacon receivers.
The design allows for drop ship deployments with 2-4 hours of field setup time on a pre-prepared site.
The camera aperture will last several years before reapplication of the heavy duty outdoor caulk is warranted. 
The cabinets can be placed off PFRR, on commercial property or a school, since they are robust against casual non-malicious encounters.
The integrated 4G modem allows automatic status updates and remote retrieval of high-interest video segments at low cost.

\section{Future Work}
HiST Phase 2 cabinets are complete and awaiting installation of the camera hardware.
Camera housings have been made for the the HiST EMCCD cameras, with a probable second EMCCD camera set up as a spectrograph in one or more cabinets.
Beacon receivers and/or HF receivers are a probable add on to each cabinet, with room for a third PC in each cabinet.
The idea behind having a separate PC per camera is to avoid random errors due to overloading of PC I/O busses.
Lab testing has shown the laptop CPUs inside highly compact computers such as the Intel NUC have I/O limitations that prevent running cameras such as the sCMOS Neo at full frame rate relative to a PC.
This is not a hard drive issue, it's a memory/PCI Express bandwidth associated limit.
This is a task any student could work on during the spring/summer to resolve.

The auroral discrimination algorithm could be upgraded to store statistics about the auroral forms. 
Initial work has been done on vectorizing aurora via skeletonization.
Skeletonization is not a perfect process out of the box on aurora due to the optically thin nature of aurora represented in~\eqref{eq:bint}.
Yet, the statistics from such techniques might be used to estimate if an auroral structure is splitting or filamentary without resorting to data inversion on every filamentary aurora, for automated on-site data inversion.
In summer 2014 two undergraduate research assistants (Amber Baurley and Sam Chen) worked on this and had initially promising results.

It would be naturally desirable to extend the HiST data inversion algorithm to 3-D inversion.
Possible methods include using natural pixel basis \citep{semeter1998} or sinusoidal basis \citep{bjornthesis} instead of rectangular pixels for a more physical gridding.
Updating the TRANSCAR model to the current Fortran version and increasing the ease of use of the Python module to run TRANSCAR in parallel would help increase the accuracy of the physics module and increase the adoption of TRANSCAR.
TRANSCAR is one of few time-dependent particle penetration models with flux transport.
Generalizing the inversion algorithm to better incorporate ISR and GPS TEC data along the lines of \citet{semeter2016} would be beneficial for improving the accuracy of the HiST data inversion.

The PiRadar system under development in spring 2017 is sponsored by Michael Hirsch and is designed for 4-D ionospheric microstructure measurement.
In a dual to the Mahali network topology, a network of \unit[10..100]{km} spaced \$300 Red Pitaya-based software defined \unit[10]{mW} radars measure intensity, Doppler, polarization and more as a network.
The pseudorandom transmit waveform can transmit data as well as other diverse radar modulation types.
With data reduction on site using \$35 Raspberry Pi 3 coprocessors, this fog-computing network can relay data back to the Internet, avoiding the need to visit the wind, solar or grid powered sites.
A potential critical infrastructure partner has been identified that has great interest in characterizing and quantifying space weather risks to their widely-dispersed generation and long distance power transmission assets.
PiRadar is poised to become another milestone in geoscience advances showing direct benefit to national security and economic growth, a true win-win.
