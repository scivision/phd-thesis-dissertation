\section{Discussion and Conclusions}\label{sec:disc}
In-situ measurements obtained from sounding rockets have long established that intense Langmuir waves with amplitudes approaching \unit[1.2]{V/m} are common features of the auroral ionosphere. 
Langmuir waves are observed propagating nearly parallel to $B$ and occur in bursts with durations of hundreds of milliseconds \citep{mcfadden1986,boehm1984,ergun1999a,ergun1999b}.
Freja observations confirm that Langmuir waves are the strongest electrostatic waves at altitudes $\sim \unit[1500]{km}$ \citep{stasiewicz1996}.
Such Langmuir waves have been suggested to play an important role in the auroral ionosphere energy flow chain, facilitating energy transfer from AW to the bulk plasma \citep{stasiewicz1996}.

Although observation of intense Langmuir waves in the ionosphere is not a new finding, the significance of ISR detected Langmuir turbulence lies in the specific wavenumber at which ISR-detectable wave activities are enhanced. 
The detection of Langmuir waves at relatively high wave numbers: $k \sim \unit[19]{m^{-1}}$ for PFISR and $k \sim \unit[39]{m^{-1}}$  for EISCAT puts constraints on the dynamics of interactions and/or on the characteristics of the underlying energy source for the enhanced waves \citep{akbari2014}.
Furthermore, there seems to be discrepancies between ISR and in-situ observations:
\begin{enumerate}
    \item \textit{in situ} measurements suggest that Langmuir turbulence should be commonly observed at altitudes above 600~km, whereas the ISR echoes are seen consistently at the F-region peak around 250-300 km.
    \item ISR measurements suggest that the intense Langmuir waves generate a well-developed turbulence which leads to caviton formation and collapse, whereas signatures of caviton formation have been completely missing in the \textit{in situ} measurement literature, with the exception of the inconclusive results of \citet{boehm1984}.
    \item \textit{In situ} measurements have shown that the enhanced Langmuir waves at higher altitudes are rather monochromatic \citep{ergun1999a,ergun1999b}. 
\end{enumerate}

A portion of the apparent inconsistencies between ISR and \textit{in situ} measurements may naturally arise from the different dynamics of beam-plasma interactions expected at different altitudes (i.e. \unit[250..300]{km} versus $>\unit[600]{km}$).
However, one should be cautious about drawing a definite conclusion that the processes underlying the radar echoes are just a low-altitude extension of the beam-plasma interactions long known from \textit{in situ} measurements.

An aspect of the ISR observations that bears discussion is the source of free energy underlying the detected turbulence.
In natural plasmas a common mechanism for intensification of Langmuir waves is the bump-on-tail instability (or inverse Landau damping) which operates upon the existence of a bump (positive slope) on the reduced (one-dimensional) electron velocity distribution function. 
This bump often represents the existence of an additional population of electrons on top of the bulk electrons. 
This population could be locally produced via acceleration or energization of a portion of the local electrons, or it could consist of electrons traveling to the observer in the form of electron beams. 
In the lower ionosphere and in the absence of any known local acceleration mechanism, a bump on the distribution function may directly arise from:
\begin{enumerate}
    \item the auroral magnetospheric-origin electron beams that are accelerated at high altitudes and propagate to the lower ionosphere (with energies of hundreds of eV to a few keV or more)
    \item the secondary electron population (with energies of a few eV to $\sim \unit[100]{eV}$) that emerges as a result of collisional interactions of the primary auroral electrons with the Earth's neutral atmosphere.
\end{enumerate}
In the former case, it is crucial to make a distinction between the electron beams accelerated by inertial AW and the inverted-V electron beams accelerated by quasi-static parallel electric fields. 

Detection of Langmuir waves in the ISR plasma-line channels enables the determination of the phase velocity $v_\varphi=\omega / k$ of the Langmuir waves and consequently the energy of the electrons directly exchanging energy with the waves. 
For PFISR observations, this energy is $\sim \unit[5]{eV}$, which falls in the energy range of secondary electrons. 
This may suggest that the secondary electrons provide the energy for the turbulence. 
Secondary electrons are known to be responsible for intensification of Langmuir waves in the E-region of the auroral ionosphere \citep{nilsson1996}. 
Langmuir wave intensification in such cases is associated with the presence of a bump, itself associated with the absorption cross-section of $N_2$, in the three-dimensional distribution function which results in reduction of Landau damping and increase in the Čerenkov emission rate. 
However, numerical modelings \citep{nilsson1996} have show that this bump appears over a broad range of pitch angles and is nearly isotropic at lower altitudes of the E-region and do not produce a bump in the reduced (one-dimensional) distribution function. 
As such, the plasma remains stable and the intensification of Langmuir waves is limited. 
At higher altitudes, the secondary electron spectra further deviate from isotropic; however, the pump becomes less pronounced due to the change in the neutral gas composition.

In summary, although secondary electrons have been suggested to cause plasma instabilities involving wave modes propagating nearly perpendicular to the magnetic field lines \citep{basu1982,jasperse2013}, are generally considered stable with respect to Langmuir waves. 
An additional evidence against the secondary electrons as the energy source for the F region Langmuir turbulence echoes, exists in the ISR data; where no correlation between the E region ionization (which is proportional to the secondary electron production rate) and the Langmuir turbulence echoes is found \citep{akbari2013}. 
It is also worth mentioning that the presence of secondary electrons with power law-like distributions, not only does not lead to instability, but instead significantly weakens the turbulence via introducing enhanced Landau damping to Langmuir waves \citep{newman1994linear,newman1994nonlinear,akbari2015}. 

The problem regarding the primary energetic auroral electrons as the source of energy for the turbulence underlying the radar echoes is that the Langmuir waves that are directly in energy exchange with such energetic electrons have wave numbers far below the detecting wave numbers of the existing incoherent scatter radars and, as such are undetectable.
Assuming that the parametric decay of Langmuir waves is the main product of the beam-plasma interactions at the observation altitudes, the energy transfers to yet smaller wave numbers, via a cascade of PDIs, and further away from the radar wavenumber. 
However, it has been shown that a well developed Langmuir turbulence does transfer a small fraction of the input energy to higher wave numbers via processes such as the caviton collapse or three-wave coalescence-like interactions \citep{akbari2014,akbari2015}, and that these could be the origin of the observed ISR echoes. 
Therefore, the possibility that the primary auroral electron beams are the direct source of the turbulence may not be ruled out.

It is necessary to make a distinction between the two types of electron beams that are commonly observed in the auroral ionosphere, i.e. the inverted-V electron beams and the field-aligned electron bursts. 
The inverted-V electron beams are produced by acceleration of warm plasma sheet electrons via quasi-static parallel electric fields at the earth acceleration region. 
It is possible for the inverted V electron beams to produce a positive slope in the one-dimensional distribution function at high altitudes close to the acceleration region. 
However, in short time scales and before the beams have the chance to travel long distances, the self-generated plasma waves quickly plateau the positive slope via  quasi-linear diffusion \citep{sanbonmatsu2001}, rapidly stabilizing the distribution. 
In addition to the quasilinear diffusion, while traveling from the acceleration region toward the ionosphere, the inverted-V electron beams become subject to adiabatic evolution under the converging magnetic field limes. 
As a result of the magnetic mirror force, the field-aligned beam diffuses toward oblique and perpendicular directions \citep{maggs1981}, which although leading to intensification of oblique propagating wave modes such as the upper-hybrid and whistler modes \citep{maggs1978,kaufmann1980,maggs1981}, further stabilize the distribution against Langmuir waves. 

Sounding rocket measurements of the three-dimensional electron velocity distributions at F-region altitudes, have consistently shown that the inverted-V electron beams often have a broad, nearly isotropic plateau that despite having a positive slope in the three-dimensional distribution function, does not produce a positive slope in the reduced (one-dimensional) distribution function in near parallel directions \citep{kaufmann1978,kaufmann1980,mcfadden1986}.
Such distributions are, therefore, stable against Langmuir turbulence. 
This general rule, however, may not be correct universally. 
From the experimental point of view, given the low temporal resolution of the electron velocity distribution function measurements, the existence of transient positive slopes in time scales shorter than the measurement resolutions may not be completely ruled out. 
Also from the theoretical point of view, a number of linear (wave refraction) \citep{maggs1978} and nonlinear (parametric type instabilities) \citep{papad1974} mechanisms have been suggested to be able to limit the growth of the beam-generated electrostatic waves and, consequently, limit the quasi-linear flattening of the beam, ultimately enabling the beam to maintain its unstable features while traveling down to the F-region altitudes. 
While such considerations can not be completely ruled out, a final evidence against the inverted-V electron beams as the source of turbulence underlying the ISR echoes, lies in the ISR data itself, where no correlation between the E-region ionization enhancements (the signature of energetic electron precipitation) and the turbulence echoes are found.

In contrast to inverted-V electron beams, field-aligned electron bursts produced by acceleration of cold ionospheric electrons via parallel electric field of inertial Alfvén waves \citep{kletzing2001,semeter2008} have characteristics that make them suitable candidates as the energy source for Langmuir wave enhancement at various altitudes. 
An important characteristic of the field-aligned bursts is their velocity dispersion, where the characteristic energy of the beam decreases from a few keV to $\sim\unit[100]{eV}$ over a period of $\sim \unit[100]{ms}$ (for a description of field-aligned bursts see \citet{mcfadden1987}). 
This dispersive behavior plays a major role in maintaining the beam distribution by continuously reproducing the positive slope of the distribution function, once flattened by the quasilinear diffusion, enabling the beam to remain unstable as it travels deep into the ionosphere \citep{ergun1993,sanbonmatsu2001}. 
This same behavior has been studied in the context of Langmuir wave enhancement in the solar wind and the type III radio bursts \citep{muschietti1990}. 

\textit{In situ} measurements obtained with sounding rockets have long established the connection between intense Langmuir waves in the auroral ionosphere and dispersive field-aligned electron bursts \citep{ergun1999a,stasiewicz1996}. 
Intense Langmuir waves observed \textit{in situ} are seen to undergo various non-linear interactions \citep{boehm1987,gough1990,ergun1999b}.
The association of:
\begin{enumerate}
    \item intense Langmuir waves with ISR Langmuir turbulence echoes
    \item dispersive Alfvénic accelerated electron beams with the source of energy underlying the echoes
\end{enumerate}  
are not obvious conclusions due to a number of discrepancies between the in-situ and ISR observations. 
Making such a connection, therefore, requires additional evidence as provided in this chapter from the joint HiST-PFISR observations.
Using the events presented in this paper and supplemental material, Table \ref{tab:events} collects the evidence for Alfvénic arcs corresponding to NEIALs.
\begin{sidewaystable}\centering
    \caption{Morphologies of events detailed in the paper and supplemental material. SLT=Strong Langmuir Turbulence}
    \label{tab:events}
    \begin{tabular}{lllll}
        \toprule
        Event Time [UT] & Aurora Morphology & ISR Morphology & Plasma Morphology & Figure \\
        \midrule
%        2007-03-18 & translating frozen-in & TODO & TODO & \\ % waiting SRI data
        2007-03-23 & splitting, flaming & 30 dB SNR ion line & SLT & \\
        2011-03-01 10:06 & flaming & $\Uparrow30$ dB ion line & Streaming upflow & \ref{fig:20110301a}, \ref{fig:20110301b} \\
%        2011-03-02 & & &  & \\
%        2012-11-07 & & & & \\
%        2013-04-01 & & & & \\
%        2013-04-04 & & & & \\
%        2013-04-07 & & & & \\
%        2013-04-11 & & & & \\
        2013-04-14 08:26 & Kinked, vortical & quiet & Thermal & \ref{fig:20130414T0826} \\
        2013-04-14 08:54 & Splitting/shedding & $\Uparrow30$ dB ion \& plasma lines & SLT & \ref{fig:20130414T0854a}, \ref{fig:20130414T0854b} \\
%        2013-04-14 09:27 & dark patches and flickering & quiet & Thermal & \ref{fig:20130414T0927}\\
%        2013-04-20 & & & & \\
%        2013-04-23 & & & & \\
%        2013-04-26 & & & & \\
%        2013-05-01 & & & & \\
        \bottomrule
    \end{tabular}
\end{sidewaystable}
The initial observations of the limited set of aurora and ISR turbulence presented suggests the following correspondence:
\begin{enumerate}
    \item inverted-V $\impliedby$ kinked aurora (Figure~\ref{fig:cartoonmorph}(a))
    \item strong Langmuir turbulence (F region) $\leftrightarrows$ splitting aurora (Figure~\ref{fig:cartoonmorph}(b))
    \item streaming upflow ($> 600$~km altitude)  $\leftrightarrows$ flaming aurora (Figure~\ref{fig:cartoonmorph}(c))
    
\end{enumerate}

\subsection{Future experimental configuration}\label{sec:future}
Using the BG3-filtered broadband prompt emission lines and bands captured by HiST cameras \citep{hirsch2016}, estimates of primary electron differential number flux with time scales fast enough to capture DAW behavior can be imaged and quantified.
Since the imaging chip provides a weighted sum of all emissions passing through the BG3 filter, spectral information is lost.
Existing spectrometers and hyper-spectral imagers \citep{goenka2016} typically have frame rates on the order of seconds to minutes.
The latest generation of EMCCD technology has $1024 \times 1024$ pixels at \unit[25]{fps}, nearly as fast as HiST imagers. 
A spectrograph is being tested for field deployment next year, colocated with HiST and with a similar FOV to resolve specific auroral emission lines. 
This additional data will help quantify the chemistry responsible for a discrete arc viewed by HiST, yielding further insights into the kinetics responsible for NEIALs.