\section{Conclusions}\label{sec:concl}
This chapter shows results from a regularization scheme using the physics encapsulated in TRANSCAR modeled eigenprofiles in a two camera simulation, with testing extended to three cameras for future 3-D work.
We observed estimates of the peak differential number flux $\hat{\Phi}_{top,0}(B_{\perp,0},E_0)$ for an auroral arc in the common FOV of the cameras, with typical error less than 30\% for auroral arcs within $2.5^\circ$ from camera boresight on magnetic zenith. 
TRANSCAR is used in a linear basis expansion of log-spaced energy bins across an energy range observed in the most common auroral events, enabling future extension to incorporate incoherent scatter radar and other instruments to form a meta-instrument for observing the ionospheric short term and long term trends.
This basis expansion is used to regularize the poorly observed vertical dimension, simultaneously enabling  high spatial resolution in $B_\perp$, which is important for capturing the detail in dynamic dispersive auroral events with \unit[10]{ms} temporal scales.
The performance estimates of this feasibility study show that a two camera system at the Poker Flat Research Range with \unit[3]{km} camera separation can give new science insights on multiple fronts, including the highly dynamic electron beam structures driving into the ionosphere.
Specifically, we can estimate the characteristic energy and $B_\perp$ peak location of the differential number flux $\Phi_{top}$.
The new observation techniques include use of filtered broadband optical emissions to select only prompt emissions with fast, highly sensitive EMCCD cameras, enabling the use of high frame rates with cadence of order \unit[10]{ms}.
The modeled HiST instrument is shown to be capable of high resolution electron precipitation characteristic energy estimates along $B_\perp$ within suitable error bounds, while retaining the qualitative morphology of the differential number flux in the spatial and energy domains.
Future work includes extending this estimate to 3-D by utilizing 3-D phantoms in the forward model and 3-D inversion of the 2-D pixel intensity images from the cameras, along with a 3 camera phase II HiST deployment to Poker Flat Research Range for a multi-year autonomous deployment beginning in the 2017 auroral season.
