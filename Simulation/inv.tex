\section{HiST Data Inversion}\label{sec:inv}

To estimate the characteristics of the time-dependent differential electron number flux $\Phi_{top}$ high in the ionosphere where collisionless processes dominate we employ a physics-based regularization scheme. 
The poorly observed $B_\parallel$ dimension is regularized with a linear basis expansion of volume emission rate eigenprofiles calculated by the TRANSCAR model.
The 33 log-spaced energy bins from \unit[58]{eV} to \unit[17.7]{keV} each have a coefficient estimated by our inversion algorithm for each $B_\perp$ location, comprising $\hat{\Phi}_{top}(B_\perp,E)$.
For the simulations of chapter~\ref{chapter:sim}, $\hat{\Phi}_{top}$ has dimensions $(219,33)$.
Regularization along $B_\parallel$ is key to finding a physically plausible solution from the infinitely many possible solutions due to the large null space of the inverse problem.
The data inversion process is outlined in the right column of Fig.~\ref{fig:overall}.
As observed in the middle row of Fig.~\ref{fig:est1dflame}, the volume emission rate is a smooth function of differential number flux and altitude.
The smoothness justifies describing the relationship between the unobservable \textit{in situ} physics and the observable auroral intensity by the Fredholm Integral of the First Kind,
%
\begin{equation}\label{eq:fiefk}
g(s) = \int_a^b K(s,t)f(t)\textrm{d}t
\end{equation}
%
where $f(t)$ is the unknown quantity, $g(s)$ is the observed quantity, and $K(s,t)$ is the kernel through which $g(s)$ is observed, encompassing optical filters, line integration of volume emission rate, and noise. 
For the present auroral tomography problem, we incorporate TRANSCAR eigenprofiles
\begin{equation}\label{eq:eigentranscar}
T(E,z) = \int_{\lambda} p_\lambda(E,z) M(\lambda) \textrm{d}\lambda
\end{equation}
in a representation of total auroral volume emission rate as
%
\begin{equation}\label{eq:fiefkE}
P(z) = \int_0^\infty T(E,z) \Phi_{top}(E)\textrm{d}E
\end{equation}
%
which has the same form as~\eqref{eq:fiefk} and may be discretized in matrix form,
%
\begin{equation}\label{eq:fiefkEmat}
\mathbf{P} = \mathbf{T}\Phi_{top}
\end{equation}
%
The discretized forms are convenient for computer implementation since the continuous integration~\eqref{eq:fiefkE} is represented by matrix multiplication~\eqref{eq:fiefkEmat}.
The BG3 filtered and atmosphere attenuated continuüm of wavelengths is observed at the camera as grayscale intensity
%
\begin{equation}\label{eq:master}
\textbf{L}\textbf{T}\Phi_{top}=\textbf{I}
\end{equation}
resulting in the data numbers $\mathbf{I}$ of~\eqref{eq:dn}.

In general $\mathbf{L}$ and $\mathbf{T}$ are not square, so the inverse $\mathbf{L}^{-1}$ and $\mathbf{T}^{-1}$ are not defined in this underdetermined system.
The ill-conditioned and Hadamard ill-posed nature of the problem arises both from the extreme problem geometry and that there is not a unique tomographic solution for the incident number flux causing an auroral display.
A method for solving such problems via brute force is the use of minimization algorithms \citep{semeter1997}. 
The algorithm selected for this effort is the Limited Memory Broyden-Fletcher-Goldfarb-Shanno algorithm \citep{Byrd1995,Zhu1997,Morales2011}, known as L-BFGS-B \citep{scipy}.
This algorithm was selected based on fast convergence for the large number $(>7000)$ of $\hat{\Phi}_{top}(B_\perp,E)$  parameters to minimize based on an empirical comparison with other contemporary minimization techniques.

For a particular realization of geophysical conditions and choice of differential number flux energy bins $\mathbf{T}$ is obtained from an off-line computation. 
As implicit in \eqref{eq:fiefk}, $\mathbf{T}$ is identical in the forward model and data inversion.
The 1-D slices of the synchronized images from each camera are stacked in column-major vector $\mathbf{I}$. 
The 2-D array $\hat{\Phi}_{top}(B_\perp,E)$ has rows arranged by precipitation energy in eV and columns arranged by $B_\perp$ location in kilometers.
We use L-BFGS-B minimization function
\begin{equation}\label{eq:bmin}
\hat{\Phi}_{top}(B_\perp,E) = \argmin_\Phi||\mathbf{I}-\mathbf{LT}\hat{\Phi}_{top} ||_2 =  \argmin_\Phi ||\mathbf{I}-\hat{\mathbf{I}}||_2
\end{equation}
with the bounds $\Phi \in \left[0,\infty\right)$ is given an initial guess $\Phi_{top}(B_\perp,E) \equiv 0$, and is allowed to run for 10--20 cycles. 
Automated measurements of $E_0$ and $B_{\perp,0}$ are accomplished via a 2-D Gaussian fitter algorithm originally based on MINPACK \citep{Minpack}. 
The region of the maximum differential number flux is fitted with a 2-D Gaussian using a Levenberg-Marquardt least squares algorithm to find the parameters best fitting the peak vicinity of $\hat{\Phi}_{top}$. 
In general, the forward model will have limitations in absolute accuracy with regard to the physical world due to the model assumptions and simplifications necessary to get a tractable computation within time and memory constraints.
We now examine simulations of two types of highly dynamic auroral events to show the feasibility of the HiST system for estimating $E_0$ and $B_{\perp,0}$.

%\section{ISR Data Inversion}\label{sec:isrinv}
Recent work on inversion of auroral observations to estimate the incident precipitation energy flux distribution at the top of the ionosphere includes \citet{zett2007,semeter2012,dahlgren2013,wedlund2013,hirsch2016}. 
An overview of the inversion process needed is
\begin{equation}\label{eq:fiefk1}
\vec{q} = \mathbf{A}\vec{\phi}
\end{equation}
where $\vec{q}$ is a $z \times 1$ column vector of ionization rates--ionization caused by electron precipitation into the ionosphere, and
\begin{equation}\label{eq:fiefk2}
\vec{P} = \mathbf{B} \vec{\phi}
\end{equation}
where $\vec{P}$ is a $z \times 1$ column vector of VER caused by kinetic interactions of precipitating electrons with the ionosphere \citep{wedlund2013}.

The electron precipitation is represented by $\vect{\phi}$, an $E \times 1$ column vector of ionization rates. 
$z$ is the number of altitude bins chosen in the discrete problem. 
$E$ is the number of differential energy bins chosen in the discrete problem. 
$\mathbf{A}$ and $\mathbf{B}$ are each of dimensions $z \times E$.





\citet{partamies2004} omits secondary electrons, since they have energies of less than 100 eV. % [p. 1968 last paragraph]
A significant portion of energy observed by the \unit[32]{eV}..\unit[30]{keV} DMSP particle detector is above \unit[8]{keV} \citep{partamies2004}. % [p.1969 first para].
p. 1964 $$ 1\textrm{mW/m}^2 = 6.24\times10^{12} \textrm{keV/m}^2\textrm{s} $$

\citet{partamies2004} used SPECTRUM, a non-standard FORTRAN 77 program translated by Annika Olsson to Matlab to calculate energy fluxes from the $N_e$ measured by ISR.
\url{http://www.lunduniversity.lu.se/lucat/user/54539ff9da4c67bb5baee00c09f22816}
\url{ftp://ftp.irf.se/pub/perm/ESRAD/SPECTRUM/}


