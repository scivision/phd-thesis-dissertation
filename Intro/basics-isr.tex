\FloatBarrier
\subsection{Joint ISR-Optical Analysis}\label{sec:isrbasic}
Phased-array incoherent scatter radars enable sampling of arbitrarily arranged beam patterns, sampled with nearly instantaneous switching between beam angular position.
PFISR has thus been used for a wide variety of auroral and ionospheric studies, given its location under the auroral oval and within reach of the southern boundary of northern polar cap activity.
The model-based iterative reconstruction (MBIR) from the HiST system has infrequent opportunities for independent confirmation from on-orbit sensors during the sub-second events of interest.
Even with an \textit{in situ} sensor available in the form of a rocket or satellite in just the right place during a sub-second event, the space-time ambiguity of any \textit{in situ} sensor attempting to measure a highly spatiotemporally dynamic event is unacceptably large.

The purpose of siting an instrument such as HiST near PFISR is about more than confirmation--the ionization production by the intense spray of magnetospheric electrons into the ionosphere creates plasma density gradients.
These gradients themselves create measurable radar backscatter, and when instabilities grow, the backscatter can grow to 1000 or more times greater strength than the quiescent conditions just tens of milliseconds before.
By estimating the precipitation characteristics above the ionosphere where kinetic interaction are minuscule compared to inside the ionosphere, the turbulence measured by the radar can be directly connected to its source.
The fine scale structure and growth of plasma instabilities on these spatiotemporal scales have been simulated \citep{akbari2015}, but confirmation of these models comes about through detailed measurements of the natural laboratory provided by Earth, HiST and PFISR as performed in chapter~\ref{chapter:fusion}.

\subsubsection{Objectives related to joint ISR-Optical Analysis}
The objectives of this dissertation with regard to joint ISR-Optical analysis are mainly accomplished via MBIR. 
MBIR on HiST high-speed synchronized video streams reveals the spatiotemporal structure of precipitation driving streaming instabilities and strong Langmuir turbulence.
Zakharov simulations \citep{akbari2015,zakharov1d} reveal the nature of instabilities developed from strong lower energy precipitation and their modeled ISR spectrum.
HiST provides high time resolution estimates of that spectrum, allowing fine scale plasma turbulence model validation.