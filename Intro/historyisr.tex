\subsection{History of Auroral Radio Science and Radar}\label{sec:historyisr}
As scientists scrambled to explain Marconi's improbable December 12, 1901 \citep{ratcliffe1974,belrose2001} \unit[3500]{km} transatlantic communications demonstration, early nineteenth century hypotheses and theory \citep{gauss1839} on the ionosphere were revived.
\citet{appleton1925} via careful observation iterated ionospheric theory and proved the bending of electromagnetic waves in the ionosphere.
Citizen scientists, radio amateurs, business and military users have benefited from thoughtful exploitation of frequency-dependent ionospheric refraction.

Tactical rockets freed by the end of World War II were pressed into service for geospace science almost immediately thereafter \citep{schmerling1966}.
The International Geophysical Year (IGY) of 1957-1958 opened orbital space to humanity, first with two Sputnik launches by the USSR.
The third anthropogenic satellite named Explorer 1 provided key information confirming the nature of the radiation belts persistence around Earth.
Before the availability of ISR and other specialized radars to study many simultaneous ionospheric plasma parameters, innovative closely spaced networks of receivers used radio stars and tailored transmissions to uncover apparent auroral $B_\perp$ velocities up to an order of magnitude faster than bulk plasma motion \citep{briggs1954}.
Bill Gordon's seminal monograph \citep{gordon1958} on incoherent scatter radar (ISR) theory led to several large ISRs being constructed within a decade.

Current ISR technology includes ``dish'' antennas such as at Arecibo, Millstone Hill and Søndrestrøm as well as several phased array radars including Jicamarca \citep{hysell2013jica}, EISCAT and AMISR.
Cubesats have been used to form a bistatic radar receiver for ISR, sensing \unit[0.1..1]{m} scale $B$-field aligned irregularities that PFISR is unable to resolve in monostatic mode \citep{bahcivan2014,cutler2013rax}.
Better spatiotemporal phased-array ISR methods have been modeled and simulated to better use scarce ISR resources.
For example, optimizing ISR beam pattern in the vicinity of a satellite while maintaining the overall observation pattern during a LEO pass \citep{swobodathesis}.
Networks of HF radars around each pole \citep{chisham2007} and growing ISR coverage fused with other sensors such as Cubesats and GPS TEC networks \citep{semeter2016} unlock increasingly fine spatiotemporal scales via synchronized simultaneous observation and inversion.
